\subsection{Defect Prediction}
\label{sec:defect}
\input{defect}

The software defect prediction has been studied in the past decade~\cite{nam2013transfer, menzies2010defect, menzies2007data, zimmermann2007predicting, jiang2013personalized, nagappan2007using, nguyen2011topic, wang2012compressed}. However, the traditional approaches in defect prediction often manually extract features from historical defect data to construct machine learning classification model~\cite{menzies2010defect}. 
%McCabe et al.~\cite{mccabe1976complexity} introduced a graph-theoretic complexity measure for the control program elements which can be considered as a feature in defect prediction. CK features~\cite{chidamber1994metrics} focused on understanding of software development process, while MOOD features~\cite{harrison1998evaluation} provided an overall assessment of a software system to manage the software development projects. These features are widely used in defect prediction. 
Moser et al.~\cite{moser2008comparative} employed the number of revisions of a file, age of a file, number of authors that checked a file, etc. for defect prediction. Nagappan et al.~\cite{nagappan2007using} extracted features by considering relationship between its software
dependencies, churn measures and post-release failures to build a classification model for defect prediction. Lee et al.~\cite{lee2011micro} introduced 56 novel micro interaction metrics (MIMs) leveraging developers' interaction information stored in the Mylyn data, and shown that MIMs significantly improve the performance of defect classification. Jiang~\cite{jiang2013personalized} showed that individual characteristics and collaboration between developers were useful for defect prediction. 

Based on these features, classification models are built to predict the defect among program elements. Elish et al.~\cite{elish2008predicting} estimated the capability of Support Vector Machine (SVM)~\cite{suykens1999least} in predicting defect-prone software modules and showed that the prediction performance of SVM is generally better than eight statistical and machine learning models in NASA datasets. Amasaki et al.~\cite{amasaki2003bayesian} employed Bayesian belief network (BBN)~\cite{mcabeebayesian} to predict the amount of residual faults of a software product. Khoshgoftaar et al.~\cite{khoshgoftaar2002tree}
showed that the Tree-based machine learning algorithms are efficiently in defect detection. Jing et al.~\cite{jing2014dictionary} proposed to use the dictionary learning technique to predict software defect. Typically, they introduced a cost-sensitive discriminative dictionary learning (CDDL) approach for software defect classification and prediction.

The main differences between our approach and traditional approaches are as follows. First, existing approaches to defect prediction are based on manually encoded traditional features which are not sensitive to programs' semantic information, while our approach automatically learns semantic features using semi-supervised autoencoder. Second, these features are automatically employed to construct classification model for defect prediction tasks. 

%The software defect prediction has been studied in the past decade~\cite{nam2013transfer, menzies2010defect, menzies2007data, zimmermann2007predicting, jiang2013personalized, nagappan2007using, nguyen2011topic, wang2012compressed}. However, the traditional approaches in defect prediction often manually extract features from historical defect data to construct machine learning classification model~\cite{menzies2010defect}. McCabe et al.~\cite{mccabe1976complexity} introduced a graph-theoretic complexity measure for the control program elements which can be considered as a feature in defect prediction. CK features~\cite{chidamber1994metrics} focused on understanding of software development process, while MOOD features~\cite{harrison1998evaluation} provided an overall assessment of a software system to manage the software development projects. These features are widely used in defect prediction. Moser et al.~\cite{moser2008comparative} employed the number of revisions of a file, age of a file, number of authors that checked a file, etc. to defect prediction. Nagappan et al.~\cite{nagappan2007using} extracted features by considering relationship between its software
%dependencies, churn measures and post-release failures to build classification model for defect prediction. Lee et al.~\cite{lee2011micro} introduced 56 novel micro interaction metrics (MIMs) leveraging developers' interaction information stored in the Mylyn data, and shown that MIMs significantly improve the performance of defect classification. Jiang~\cite{jiang2013personalized} showed that individual characteristics and collaboration between developers were useful for defect prediction. 
%
%Based on these features, classification models are built to predict the defect among program elements. Elish et al.~\cite{elish2008predicting} estimated the capability of Support Vector Machine (SVM)~\cite{suykens1999least} in predicting defect-prone software modules and showed that the prediction performance of SVM is generally better than eight statistical and machine learning models in NASA datasets. Amasaki et al.~\cite{amasaki2003bayesian} employed Bayesian belief network (BBN)~\cite{mcabeebayesian} to predict the amount of residual faults of a software product. Khoshgoftaar et al.~\cite{khoshgoftaar2002tree}
%showed that the Tree-based machine learning algorithms are efficiently in defect detection. Jing et al.~\cite{jing2014dictionary} proposed to use the dictionary learning technique to predict software defect. Typically, they introduced a cost-sensitive discriminative dictionary learning (CDDL) approach for software defect classification and prediction.
%
%The main differences between our approach and traditional approaches are as follows. First, existing approaches to defect prediction are based on manually encoded traditional features which are not sensitive to programs' semantic information, while our approach automatically learns semantic features using semi-supervised autoencoder. Second, these features are automatically employed to construct classification model for defect prediction tasks. 

%\textcolor{red}{Talking about the different between our approach with DBN approach}
%
%The main dierences between our approach and existing
%approaches for within-project defect prediction and cross-project defect prediction are as follows. First, existing ap-proaches to defect prediction are based on manually encoded
%traditional features which are not sensitive to programs' se-mantic information, while our approach automatically learns
%semantic features using DBN and uses these features to per-form defect prediction tasks. Second, since our approach re-quires only the source code of the training and test projects,
%it is suitable for both within-project defect prediction and
%cross-project defect prediction.

%many machine learning models
%are built for two dierent defect prediction tasks|within-project defect prediction and cross-project defect prediction.
 

%ages of les as features to predict defects. Nagappan et
%al. [40] proposed code churn features, and shown that these
%features were eective for defect prediction. Hassan et
%al. [12] used entropy of change features to predict defects.
%Lee et al. [27] proposed 56 micro interaction metrics to
%improve defect prediction. Other process features, including
%developer individual characteristics [18, 48] and collabora-tion between developers [27, 34, 51, 64], were also useful for
%defect prediction.

%Within-project defect prediction (WPDP) uses training
%data and test data that are from the same project. Many
%machine learning algorithms have been adopted for WPDP,
%including Support Vector Machine (SVM) [8], Bayesian
%Belief Network [1], Naive Bayes (NB) [59], Decision Tree
%(DT) [9, 21, 62], and Dictionary Learning [20].
%Elish et al. [8] evaluated the capability of SVM in predict-ing defect-prone software modules, and they compared SVM
%against eight statistical and machine learning models on four
%NASA datasets. Amasaki et al. [1] proposed an approach to
%predict the nal quality of a software product by using the
%Bayesian Belief Network. Tao et al. [59] proposed a Naive
%Bayes based defect prediction model, they evaluated the pro-posed approach on 11 datasets from the PROMISE defect
%data repository. Wang et al. [62] and Khoshgoftaar et al. [21]
%examined the performance of Tree-based machine learning
%algorithms on defect prediction, their results suggested that
%Tree-based algorithms could help defect prediction. Jing et
%al. [20] introduced the dictionary learning technique to de-fect prediction. They proposed a cost-sensitive dictionary
%learning based approach to improve defect prediction.

\subsection{Deep Learning in Software Engineering}
\label{sec:deeplearning}
Deep learning algorithms have been widely used to improve research tasks in software engineering in the past few years. Lam et al.~\cite{lam2015combining} combined deep neural network (DNN)~\cite{hecht1988theory} with rVSM~\cite{zhou2012should}, a revised vector space model, to improve the performance of bug localization. Raychev et al.~\cite{raychev2014code} reduced the problem ofa code completion problem to a natural-language processing problem of predicting sentences' probabilities of sentences and. They used recurrent neural network~\cite{mikolov2010recurrent} to predict the probabilities of the next tokensubsequent words in a sentence. Mou et al.~\cite{mou2014tbcnn} proposed a tree-based convolutional neural network (TBCNN) for programming language processing. Results of their experiment showed that the effectiveness of TBCNN  in two different program analysissoftware engineering tasks: classifying programs according to functionality, and detecting code snippets of certain patterns. Pascanu et al.~\cite{pascanu2015malware} employed recurrent neural network to build a malware classification model in software system. Yuan et al.~\cite{yuan2014droid} adopted deep belief network (DBN)~\cite{hinton2009deep} to predict mobile malware in Android platform. Their experimental results showed that a deep learning technique is especially suitable for predicting malware in software system.                            

%Yang et al.~\cite{yang2015deep} leveraged DBN to generate features from existing features and used these new features to predict whether a program element contains bugs. It showed that the deep learning algorithm helps to discover more bug than tradition approaches on average across from six large software projects. The existing features were manually designed based on change level: i.e., the number of modified subsystems, code added, code deleted, the number of files change, etc. In 2016, Wang et al.~\cite{wang2016automatically} also employed DBN to learn semantic features from source code. However, the existing features were extracted from abstract syntax tree since~\cite{wang2012compressed} claimed that Yang features~\cite{yang2015deep} were fail to distinguish the semantic difference among source code. The evaluation on ten popular source projects showed that the semantic features significantly improved the performance of defect detection. Different to the existing works that semantic features and defect prediction model are built independently, thus the semantic features only learn from source code without considering the label of this program element which may decrease the performance of defect prediction model. To tackle this problem, we propose a deep  discriminative autoencoder to build classification model for solving defect prediction problem.  We evaluate the effectiveness of our proposed approaches against Wang approaches~\cite{wang2012compressed} and the traditional machine learning algorithms (i.e., naive bayes, logistic regression, and random forest) on four popular software projects .

%Our work diers from the above study mainly in three
%aspects. First, we use DBN to learn semantic features directly
%from source code, while features generated from their
%approach are relations among existing features. Since the existing
%features used cannot distinguish many semantic code
%dierences, the combination of these features would still fail
%to distinguish the semantic dierences. For example, if two
%changes add the same line at dierent locations in the same
%le, the traditional features used cannot distinguish the two
%changes. Thus, the generated new features, which are combinations
%of the traditional features, would also fail to distinguish
%the two changes. Second, we evaluate the eectiveness
%of our generated features using dierent classiers and
%for both within-project and cross-project defect prediction,
%while they use LR only for within-project defect prediction.
%Third, we focus on le level defect prediction, while they
%work on change level defect prediction.

%Yang et
%al. [68] proposed an approach that leveraged deep learning
%to generate features from existing features and then used
%these new features to predict whether a commit is buggy or
%not. This work was motivated by the weaknesses of logistic
%regression (LR) that LR can not combine features to generate
%new features. They used DBN to generate features from
%14 traditional change level features: the number of modi
%ed subsystems, modied directories, modied les, code
%added, code deleted, line of code before/after the change,
%les before/after the change, and several developer experience
%related features [68]. 

%Other studies leverage deep learning to address other
%problems in software engineering. Lam et al. [26] combined
%deep learning algorithms and information retrieval tech-niques to improve fault localization. Raychev et al. [53]
%reduced the code completion problem to a natural language
%processing problem and used deep learning to predict the
%probabilities of next tokens. White et al. [65] leveraged deep
%learning to model program languages for code suggestion.
%Similarly, Mou et al. [39] used deep learning to model
%programs and showed that deep learning can capture
%programs' structural information. In addition, deep learn-ing has also been used for malware classication [50, 69],
%acoustic recognition [24,36,37], etc.

