%In this paper, WHYPER framework using Natural Language Processing (NLP) techniques to determine why an application uses a permission is proposed. The framework is evaluated on real-world application descriptions that involve three permissions (address book, calendar, and record audio).  The experimental results show that that WHYPER achieves an average precision of 82.8\%, and an average recall of 81.5\% for three permissions. It proves that we can use NLP techniques to bridge the semantic gap of user expectations to aid the risk assessment of mobile applications.

Our paper presents a deep semi-supervised learning to optimize defect prediction model. Typically, we take advantage of deep learning autoencoder to learn semantic features from token vectors extracted from programs' ASTs automatically, and optimize these feature to construct classification model for predicting defects. Our evaluation on four software projects shows that our approaches could significantly improve the performance of defect prediction compared to two traditional approaches, i.e., AST features and semantic features. In the future, we would like to extend our automatically semantic feature generation approach to C/C++ projects for defect prediction. In addition, it would be promising to leverage our approach to automatically build defect prediction model at different levels, i.e., change level, module level, or package level, etc. instead of file level. 

%generate features for predicting defects at other levels, such as change level, module level, and package level.
%
%This paper proposes to leverage a representation-learning
%algorithm, deep learning, to learn semantic representation
%directly from source code for defect prediction. Specically,
%we deploy Deep Belief Network to learn semantic features
%from token vectors extracted from programs' ASTs automatically,
%and leverage the learned semantic features to build
%machine learning models for predicting defects.
%Our evaluation on ten open source projects shows that
%the automatically learned semantic features could signi-
%cantly improve both within-project and cross-project defect
%prediction compared to traditional features. Our semantic
%features improve the within-project defect prediction on average
%by 14.7% in precision, 11.5% in recall, and 14.2% in
%F1 comparing with traditional features. For cross-project
%defect prediction, our semantic features based approach improves
%the state-of-the-art technique TCA+ built on traditional
%features by 8.9% in F1.
%In the future, we would like to extend our automatically
%semantic feature generation approach to C/C++ projects
%for defect prediction. In addition, it would be promising
%to leverage our approach to automatically generate features
%for predicting defects at other levels, such as change level,
%module level, and package level.
