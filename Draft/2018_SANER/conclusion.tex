%In this paper, WHYPER framework using Natural Language Processing (NLP) techniques to determine why an application uses a permission is proposed. The framework is evaluated on real-world application descriptions that involve three permissions (address book, calendar, and record audio).  The experimental results show that that WHYPER achieves an average precision of 82.8\%, and an average recall of 81.5\% for three permissions. It proves that we can use NLP techniques to bridge the semantic gap of user expectations to aid the risk assessment of mobile applications.

This paper presents a new deep discriminative autoencoder (DDA) approach to achieve an effective software defect prediction. DDA provides an end-to-end learning approach to simultaneously learn embedding features that can well represent token vectors extracted from programs' ASTs, and build an accurate classification model for defect prediction. Empirical studies on four software projects show that our approach significantly outperforms the existing defect prediction approaches. Specifically, our approach improve the F1 score by 19.63\% and 18.95\% when compared with the state-of-the-art approach for both within-project and cross-project setting, respectively. 

While DDA offers a powerful approach for defect prediction, there remains room for improvement. In the future, we plan to improve the effectiveness of our approach, potentially by either generating better features or enhancing our DDA model. We also plan to experiment on more dataset and evaluate our approach more thoroughly. 

%For example, DDA currently takes as input the TF-IDF features constructed from AST nodes, thus ignoring the meaning of AST nodes within source code of a program. In the future, we wish to learn the semantics of AST nodes by employing word embedding representation. We also plan to develop a more sophisticated deep learning method to better capture sequence information within a defect prediction task. 

%In the future, we wish to learn meaning of AST nodes by taking advantage of word vector 
%
%In the future, we would like to extend our automatic semantic feature generation approach to support defect prediction in C/C++ projects. In addition, it would be promising to leverage our approach to automatically build defect prediction model at different level (i.e., change level, module level, or package level, etc.) instead of file level. 

%generate features for predicting defects at other levels, such as change level, module level, and package level.
%
%This paper proposes to leverage a representation-learning
%algorithm, deep learning, to learn semantic representation
%directly from source code for defect prediction. Specically,
%we deploy Deep Belief Network to learn semantic features
%from token vectors extracted from programs' ASTs automatically,
%and leverage the learned semantic features to build
%machine learning models for predicting defects.
%Our evaluation on ten open source projects shows that
%the automatically learned semantic features could signi-
%cantly improve both within-project and cross-project defect
%prediction compared to traditional features. Our semantic
%features improve the within-project defect prediction on average
%by 14.7% in precision, 11.5% in recall, and 14.2% in
%F1 comparing with traditional features. For cross-project
%defect prediction, our semantic features based approach improves
%the state-of-the-art technique TCA+ built on traditional
%features by 8.9% in F1.
%In the future, we would like to extend our automatically
%semantic feature generation approach to C/C++ projects
%for defect prediction. In addition, it would be promising
%to leverage our approach to automatically generate features
%for predicting defects at other levels, such as change level,
%module level, and package level.
